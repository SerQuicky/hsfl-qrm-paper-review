\documentclass{hsflensburg}

% packages
%-------------------------------------------
\usepackage[utf8]{inputenc}
\usepackage[ngerman]{babel}
\usepackage{csquotes}
\usepackage{placeins}
\usepackage{url}
\usepackage{xcolor}

\MakeOuterQuote{"}
{\renewcommand{\arraystretch}{2.0}% for the vertical padding

\definecolor{orange}{HTML}{FF7F00}

% document title
%-------------------------------------------
\title{Paper-Review}
\subtitle{Qualitative Research Methods}

% authors
%-------------------------------------------
\author{
	\name{Tom Hartelt}\\
	\institution{Hochschule Flensburg}
	\and
	\name{Martin Hermannsen}\\
	\institution{Hochschule Flensburg}
	\and
	\name{Michael Frank}\\
	\institution{Hochschule Flensburg}
}

\begin{document}
	\maketitle
 	 \tableofcontents

  \pagebreak
	
	\section{Zusammenfassung}
	In der Arbeit \textit{Teenagers and Their Virtual Possessions: Design Opportunities and Issues} von
 	William Odom, John Zimmerman und Jodi Forlizzi aus dem Jahr 2011 wird die Bedeutsamkeit 
	und emotionale Bindung von Jugendlichen zu ihren virtuellen Besitztümern\footnote{ z.B. Nachrichten aus sozialen Medien, digitale Bilder oder digitale Flugtickets} untersucht.\\
	Im Rahmen der Studie wurden Semi-Strukturiertes Interviews mit  Jugendlichen zu ihren physischen und virtuellen Besitztümern 	geführt, in denen die Forschenden Einblicke in das alltägliche Leben der 
	Jugendlichen und deren Umgang mit ihren virtuellem Besitz erlangten. Außerdem wurde
	 in diesen Interviews des Öftern der Umgang mit den sozialen Medien wie bspw. Facebook thematisiert, zumal sich viele virtuelle Besitztümer der Jugendlichen auf diesen Plattformen befinden 
	\cite{odom2011teenagers}. \\

	Anhand der Interviews konnten die Forschenden feststellen, dass die 
	virtuellen Besitztümer und sozialen Medien neue Möglichkeiten der Identitätsbildung 
	und -experementierung ermöglichen. Außerdem schlagen die Forschenden auf Basis der 
	Erkenntnisse der Interviews potentielle Designentscheidungen für die 
	Entwicklung neuer Systemen vor, welche den Umgang mit virtuellen 
	Besitztümern verbessern und vereinfachen könnten \cite{odom2011teenagers}. 


	\section{Studienmethodik}
	Im Rahmen der Studie  wurden insgesamt einundzwanzig Jugendliche im Alter
	von zwölf bis siebzehn Jahren interviewt. Alle Jugendlichen stammen
	aus einer mittelgroßen Stadt in den USA. Die Testpersonen gehörten zu Familien
	der gesellschaftlichen Mittel- bzw. Oberschicht und hatten größtenteils Zugang 
	zum Internet \cite{odom2011teenagers}. \\

	Es wurden mit jeder Testperson ein ungefähr neunzig bis einhundertzwanzigminütiges
	Semi-Strukturiertes Interview in dem eigenen Schlafzimmer geführt. Im Interview
	wurden die Studienteilnehmenden gebeten, den Forschenden ihre physischen Besitztümer
	in ihrem Zimmer zu zeigen. In der Regel folgte danach das Vorzeigen der
	virtuellen Besitztümer, welcher sich meist auf dem Computer oder Smartphone befand 
	\cite{odom2011teenagers}. \\

	Die Interviews wurden mithilfe von Videoaufnahmen, Fotos und Notizen für die spätere Auswertung dokumentiert. Nach der Durchführung
	der Interviews wurden anhand der Aufnahmen und Notizen verschiedene Themen
	definiert. Mithilfe der Themen wurden die ausgearbeiteten
	Dokumente codiert. Aus den codierten Dokumenten wurden abschließend Diagramme und Modelle für eine visuelle Darstellung und Auswertung erstellt  \cite{odom2011teenagers}. 

	\section{Diskussion zur Studienmethodik}

	\subsection{Positive Anmerkungen}

	Die Forschenden haben die Interviews in den Schlafzimmern der Jugendlichen durchgeführt, um eine vertraute Studienumgebung zu schaffen \cite{odom2011teenagers}. Die Wahl des Studienorts erscheint sinnvoll gewählt, um die physischen und virtuellen Besitztümer der Testpersonen und dessen Umgang damit zu erforschen. Wir gehen davon aus, dass die Studienteilnehmenden in einer neutralen Umgebung weniger detaillierte Daten angegeben hätten, daher hat der Studienort entscheidend zu der Ermittlung der Daten beigetragen.

	Es wurden für die Studie verschiedene Auswertungsmethoden angewendet (Videos, Fotos, Notizen, Diagramme und Modelle) \cite{odom2011teenagers}. Dies bestärkt eine tiefgehende Analyse aller verfügbaren Daten und Eindrücke. Gleichwohl bleibt die Vorgehensweise bei der Analyse unbekannt, da die Auswertung der Daten nicht näher beschrieben wurde.   

	Durch die ermittelten Daten haben die Forschenden eine sinnvolle Kategorisierung in den \textit{Findings} vorgenommen (\textit{capture the emerging themes}). Die vier Themen (\textit{storage of virtual possessions; how virtual possessions are curated and displayed to manage presentation of self; how social metadata can be a crucial part of virtual possessions; and how artifacts transition between material and virtual forms}) wurden vermutlich durch die Codierung der Daten als Themenfelder herausgearbeitet. Durch wiederholte Diskussion und \textit{modeling sessions} in den \textit{Findings} leiten die Forschenden drei spezifische Forschungsgebiete für weitere Investigationen ab (\textit{accrual of metadata, placelessness and presence, and presentation of selves.})  \cite{odom2011teenagers}.   

	\subsection{Kritische Anmerkungen}

	Das Paper basiert auf einem \textit{Contextual Inquiry} Studiendesign, da die Umgebung des Schlafzimmers mit untersucht und dazu verbale Aussagen der Jugendlichen in Form eines Semi-Strukturierten Interviews aufgenommen wurden. Es wird dabei jedoch nicht ersichtlich, wie die Forschenden bei der Studie exakt vorgegangen sind. Trotz der vielfältigen Analysemethoden fehlt ein Überblick über die Datenermittlung und es bleibt weitestgehend unklar, welche Fragen in den Semi-Strukturieren Interviews gestellt wurden. Für die Schlussfolgerung der Arbeit sind diese Angaben zum Teil entbehrlich, jedoch muss die lesende Person den Forschenden dahingehend vertrauen, dass diese eine hohe Qualität bei ihren Untersuchungen eingehalten haben und alle Studienteilnehmenden gleichartig untersucht wurden.
	
	Beim Sampling der Testpersonen wurden Jugendliche im Alter von 12-17 Jahren aus sehr ähnlichen Verhältnissen und derselben Stadt ausgewählt  \cite{odom2011teenagers}. Uns erscheint dies eine sehr homogene Gruppe zu sein, wodurch die möglichen Studienteilnehmenden stark eingeregrenzt werden. Andere Personas wurden nicht untersucht, sodass die Studie nicht repräsentativ ist. Diese Limitation ist den Forschenden jedoch bewusst und sie weisen explizit darauf hin. 

	Weiterhin gehen die Forschenden intensiv auf die Aussagen in den \textit{Findings} ein und schlussfolgern daraus die Verbesserungsmöglichkeiten. Die Daten wurden scheinbar sehr interpretativ ausgewertet, da die Forschenden die \textit{Painpoints} der Testpersonen direkt aufgegriffen haben, um daraus die Möglichkeiten und negative Konsequenzen zu beschreiben.

	Bei der Durchführung der Studie wurden sehr intime Bereiche der Jugendlichen untersucht. Dabei stellt sich die Frage, ob die Forschenden die Interviews auch bei andersgeschlechtlichen Studienteilnehmenden durchgeführt haben (z.B. Mann befragt Mädchen). Dies könnte aufgrund von Schamgefühl oder fehlendem Vertrauen dazu führen, dass die physischen und virtuellen Besitztümer nicht vollständig präsentiert wurden. Gleichzeitig kann die Aufzeichnung von Videomaterial für die Jugendlichen im privaten Umfeld unangenehm sein und es ist möglich, dass nicht alle privaten Details der physischen und virtuellen Besitztümer offengelegt wurden. Zudem ist die Methode einer Videoaufnahme für die Jugendlichen präsenter beziehungsweise aufregender als eine Audioaufnahme, sodass womöglich die Daten durch die Aufregung verfälscht werden könnten. 

	Da vor der Studie bereits ein Pilotprojekt durchgeführt wurde, ist es möglich, dass die Studienteilnehmenden bereits über die Studie oder dessen Ziele informiert waren  \cite{odom2011teenagers}. Eine vorhergehende Beeinflussung kann dadurch nicht ausgeschlossen werden. 


	\clearpage
	\nocite{*}	
	\bibliography{literature}
	\bibliographystyle{abbrv}
\end{document}
