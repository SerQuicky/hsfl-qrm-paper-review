\documentclass{hsflensburg}

% packages
%-------------------------------------------
\usepackage[utf8]{inputenc}
\usepackage[ngerman]{babel}
\usepackage{csquotes}
\usepackage{placeins}
\usepackage{url}
\usepackage{xcolor}

\MakeOuterQuote{"}
{\renewcommand{\arraystretch}{2.0}% for the vertical padding

\definecolor{orange}{HTML}{FF7F00}

% document title
%-------------------------------------------
\title{Paper-Review}
\subtitle{Qualitative Research Methods}

% authors
%-------------------------------------------
\author{
	\name{Tom Hartelt}\\
	\institution{Hochschule Flensburg}
	\and
	\name{Martin Hermannsen}\\
	\institution{Hochschule Flensburg}
	\and
	\name{Michael Frank}\\
	\institution{Hochschule Flensburg}
}

\begin{document}
	\maketitle
 	 \tableofcontents

  \pagebreak
	
	\section{Zusammenfassung}
	In der Arbeit \textit{Teenagers and Their Virtual Possessions: Design Opportunities and Issues} von
 	William Odom, John Zimmerman und Jodi Forlizzi aus dem Jahr 2011 wird die Bedeutsamkeit 
	und emotionale Bindung von Jugendlichen zu ihren virtuellen Besitztümern\footnote{ z.B. Nachrichten aus sozialen Medien, digitale Bilder oder digitale Flugtickets} untersucht.\\
	Im Rahmen der Studie wurden Gespräche mit  Jugendlichen zu ihren physischen und virtuellen Besitztümern 	geführt, in denen die Forschenden Einblicke in das alltägliche Leben der 
	Jugendlichen und deren Umgang mit ihren virtuellem Besitz erlangten. Außerdem wurde
	 in diesen Interviews des Öftern der Umgang mit den sozialen Medien wie bspw. Facebook thematisiert, zumal sich viele virtuelle Besitztümer der Jugendlichen sich auf diesen Plattformen befinden 
	\cite{odom2011teenagers}. \\

	Anhand der Interviews mit den Jugendlichen konnten die Forschenden feststellen, dass die 
	virtuellen Besitztümer und sozialen Medien neue Möglichkeiten der Identitätsbildung 
	und -experementierung ermöglichen. Außerdem schlagen die Forschenden auf Basis der 
	Erkenntnisse der Interviews potentielle Designentscheidungen für die 
	Entwicklung neuer Systemen vor, welche den Umgang mit virtuellen 
	Besitztümern verbessern und vereinfachen könnten \cite{odom2011teenagers}. 


	\section{Studienmethodik}
	Im Rahmen der Studie  wurden insgesamt einundzwanzig Jugendliche im Alter
	von zwölf bis siebzehn Jahren interviewt. Alle Jugendlichen stammen
	aus einer mittelgroßen Stadt in den USA. Die Testpersonen gehörten zu Familien
	der gesellschaftlichen Mittel- bzw. Oberschicht und hatten größtenteils Zugang 
	zum Internet \cite{odom2011teenagers}. \\

	Es wurden mit jeder Testperson ein ungefähr neunzig bis einhundertzwanzigminütiges
	Semi-Strukturiertes Interview in dem eigenen Schlafzimmer geführt. Im Interview
	wurden die Studienteilnehmenden gebeten, den Forschenden ihre physischen Besitztümer
	in ihrem Zimmer zu zeigen. In der Regel folgte danach das Vorzeigen der
	virtuellen Besitztümer, welcher sich meist auf dem Computer oder Smartphone befand 
	\cite{odom2011teenagers}. \\

	Die Interviews wurden mithilfe von Videoaufnahmen, Fotos und Notizen für die spätere Auswertung dokumentiert. Nach der Durchführung
	der Interviews wurden anhand der Aufnahmen und Notizen verschiedene Themen
	definiert. Mithilfe der Themen (Codes) wurden die ausgearbeiteten
	Dokumente codiert. Aus den codierten Dokumenten wurden abschließend Diagramme und Modelle für eine visuelle Darstellung und Auswertung erstellt  \cite{odom2011teenagers}. 

	\section{Diskussion zur Studienmethodik}

	\underline{Positive Punkte}

	Die Forschenden haben die Interviews in den Schlafzimmern der Probanden durchgeführt um eine vertraute Umgebung für die Studie zu schaffen. Die Wahl des Studienorts erscheint sinnvoll gewählt um die materiellen und virtuellen Besitztümer der Probanden und dessen Umgang damit zu erforschen. Wir gehen davon aus, dass die Probanden in einer neutralen Umgebung weniger detaillierte Daten angegeben hätten, daher hat der Studienort entscheidend für die Gewinnung der Daten beigetragen.

	Es wurden für die Studie verschiedene Auswertungsmethoden genutzt (Videos, Fotos, Notizen, Diagramme und Modelle). Dies spricht für eine tiefgehende Analyse aller verfügbaren Daten und Eindrücke. Gleichwohl bleibt dem Leser die Vorgehensweise bei der Analyse unbekannt, da die Auswertung der Daten im Paper nicht näher beschrieben wurde.   

	Durch die gewonnenen Daten haben die Forschenden eine sinnvolle Kategorisierung in den \textit{Findings} vorgenommen (\textit{capture the emerging themes}). Die vier Themen (\textit{storage of virtual possessions; how virtual possessions are curated and displayed to manage presentation of self; how social metadata can be a crucial part of virtual possessions; and how artifacts transition between material and virtual forms}) sind vermutlich durch die Codierung der Daten als Themenfelder herausgearbeitet worden. Durch wiederholte Diskussion und \textit{modeling sessions} in den \textit{Findings} leiten die Forschenden drei spezifische Forschungsgebiete für weitere Investigationen ab (\textit{accrual of metadata, placelessness and presence, and presentation of selves.}).   

	\begin{itemize}
	\item Gute Struktur - Kategosierung in den Findings (Captering emerging themes)
 	\item Interviewort (Schlafzimmer) / Studienumgebung  gut gewählt, da die Besitz im Schlafzimmer ist -> lockere Stimmung
 	\item Viele Auswertungsmethoden (Video, Notizen, Diagramme, Photo und Modelle)
	\end{itemize}

	\underline{Kritische Punkte}

	Das Paper basiert auf einem Contextual Inquiry Studiendesign, da die Umgebung des Schlafzimmers mit begutachted wird und dazu verbale Aussagen der Probanden in Form eines semi-strukturierten Interviews aufgenommen wurden. Dem Leser des Papers wird dabei jedoch nicht ersichtlich, wie die Forschenden bei der Studie exakt vorgegangen sind. Trotz der vielfältigen Analysemethoden fehlt dem Leser ein Überblick über die Datengewinnung und es bleibt unklar, welche Fragen in den semi-strukturieren Interviews gestellt wurden. Für die Schlussfolgerung des Papers sind diese Angaben vielleicht entbehrlich, jedoch muss der Leser den Forschenden dahingehend vertrauen, dass diese eine hohe Qualität bei ihren Untersuchungen eingehalten haben und alle Probanden gleichartig untersucht wurden.
	
	Beim Sampling der Probanden wurden Teenager im Alter von 12-17 Jahren aus sehr ähnlichen Verhältnissen und derselben Stadt ausgewählt. Uns erscheint dies eine sehr homogene Gruppe zu sein und grenzt die möglichen Probanden von vorn herein stark ein. Andere Personas wurden nicht untersucht, so dass die Studie nicht repräsentativ ist. Diese Limitation ist den Forschern jedoch bewusst und sie weisen explizit darauf hin. 

	Die Forscher gehen stark auf die Aussagen in den \textit{Findings} ein und schlussfolgern daraus die Verbesserungsmöglichkeiten. Die Daten wurden scheinbar sehr interpretativ ausgewertet, da die Forscher die Painpoints der Probanden direkt aufgegriffen haben um daraus die Möglichkeiten und negative Konsequenzen zu beschreiben.

	Bei der Durchführung der Studie wurden sehr intime Bereiche der Teenager untersucht. Dabei stellt sich uns die Frage, ob die Forschenden die Interviews auch bei andersgeschlechtlichen Probanden durchgeführt haben (z.B. Mann befragt Mädchen). Dies könnte aufgrund von Schamgefühl oder fehlendem Vertrauen dazu führen, dass die materiellen und virtuellen Besitztümer nicht vollständig präsentiert wurden. Gleichzeitig kann die Aufzeichnung von Videomaterial für die Teenager im privaten Umfeld sehr peinlich sein und es ist möglich, dass nicht alle privaten Details der materiellen und virtuellen Besitztümer offengelegt wurden. Zudem ist die Methode einer Videoaufnahme für die Teenager tendenziell aufregender als eine reine Audioaufnahme, sodass womöglich die Daten durch die Aufregung verfälscht werden könnten. 

	Da vor der Studie bereits ein Pilotprojekt durchgeführt wurde ist es möglich, dass die Teenager bereits über die Studie oder dessen Ziele informiert waren. Eine vorhergehende Beeinflussung kann dadurch nicht ausgeschlossen werden. 

	\begin{itemize}
	\item Welche Fragen wurden gestellt, wie wurden die Interviews geführt?
	\item Kinder kommen aus sehr ähnlichen Verhältnissen und derselben Stadt = homogene Masse = schlechte Sample Size
	\item Starke Eingrenzung der Probandengruppe = nicht repräsentativ für alle Personen. Diese Limitation ist den Forschern jedoch bekannt.
	\item Forscher gehen sehr interpretativ auf die Aussagen der Probanden ein und schlussfolgern daraus Verbesserungsmöglichkeiten.
	\item Das Paper basiert auf einem "Contextual Inquiry" Studiendesign, da die Umgebung des Schlafzimmers mit begutachted wird und dazu verbale Aussagen der Probanden in Form eines semi-strukturierten Interviews aufgenommen wurden.
	\item Haben auch Männer die Interviews bei Mädchen geführt? Falls ja, ggf. weniger Vertrauen beim Vorzeigen persönlicher Gegenstände
	Gleichzeitig kann die Aufzeichnung von Videomaterial für die Teenager im privaten Umfeld sehr privat sein und es ist möglich, dass nicht alle privaten Details der materiellen und virtuellen Besitztümer offengelegt werden.  
	\item (Nicht definiert bzw. genau gesagt was die genauen Dokumente sind, die nach der Erstellung der Codes, codiert wurde? Sind
		das Transkripierte Interviews, die Notizen?)
		\item Digitale Bestiztümer meist sehr privat – wird nicht alles gezeigt evtl?
		\item Es gab Pilotprojekt. Wie gut wussten Teenager also schon vorher über das Thema bescheid? Beeinflusst?
		\item Interviewlänge - Sind die Interviews nicht zu lange, ist die Länge gerechfertigt?
		\item Videoaufnahme sehr Präsent – vielleicht aufregender bzw. anderes Verhalten als nur bei Audioaufnahme
	\end{itemize}
	

	\clearpage
	\nocite{*}	
	\bibliography{literature}
	\bibliographystyle{abbrv}
\end{document}
