\documentclass{hsflensburg}

% packages
%-------------------------------------------
\usepackage[utf8]{inputenc}
\usepackage[ngerman]{babel}
\usepackage{csquotes}
\usepackage{placeins}
\usepackage{url}
\usepackage{xcolor}

\MakeOuterQuote{"}
{\renewcommand{\arraystretch}{2.0}% for the vertical padding

\definecolor{orange}{HTML}{FF7F00}

% document title
%-------------------------------------------
\title{Paper-Review}
\subtitle{Qualitative Research Methods}

% authors
%-------------------------------------------
\author{
	\name{Tom Hartelt}\\
	\institution{Hochschule Flensburg}
	\and
	\name{Martin Hermannsen}\\
	\institution{Hochschule Flensburg}
	\and
	\name{Michael Frank}\\
	\institution{Hochschule Flensburg}
}

\begin{document}
	\maketitle
 	 \tableofcontents

  \pagebreak
	
	\section{Zusammenfassung}
	In der Arbeit \textit{Teenagers and Their Virtual Possessions: Design Opportunities and Issues} von
 	William Odom, John Zimmerman und Jodi Forlizzi aus dem Jahr 2011 wird die Bedeutsamkeit 
	und emotionale Bindung von Jugendlichen zu ihren virtuellen Besitztümern\footnote{ z.B. Nachrichten aus sozialen Medien, digitale Bilder oder digitale Flugtickets} untersucht.\\
	Im Rahmen der Studie wurden Gespräche mit  Jugendlichen zu ihren physischen und virtuellen Besitztümern 	geführt, in denen die Forschenden Einblicke in das alltägliche Leben der 
	Jugendlichen und deren Umgang mit ihren virtuellem Besitz erlangten. Außerdem wurde
	 in diesen Interviews des Öftern der Umgang mit den sozialen Medien wie bspw. Facebook thematisiert, zumal sich viele virtuelle Besitztümer der Jugendlichen sich auf diesen Plattformen befinden 
	\cite{odom2011teenagers}. \\

	Anhand der Interviews mit den Jugendlichen konnten die Forschenden feststellen, dass die 
	virtuellen Besitztümer und sozialen Medien neue Möglichkeiten der Identitätsbildung 
	und -experementierung ermöglichen. Außerdem schlagen die Forschenden auf Basis der 
	Erkenntnisse der Interviews potentielle Designentscheidungen für die 
	Entwicklung neuer Systemen vor, welche den Umgang mit virtuellen 
	Besitztümern verbessern und vereinfachen könnten \cite{odom2011teenagers}. 


	\section{Studienmethodik}
	Im Rahmen der Studie  wurden insgesamt einundzwanzig Jugendliche im Alter
	von zwölf bis siebzehn Jahren interviewt. Alle Jugendlichen stammen
	aus einer mittelgroßen Stadt in den USA. Die Testpersonen gehörten zu Familien
	der gesellschaftlichen Mittel- bzw. Oberschicht und hatten größtenteils Zugang 
	zum Internet \cite{odom2011teenagers}. \\

	Es wurden mit jeder Testperson ein ungefähr neunzig bis einhundertzwanzigminütiges
	Semi-Strukturiertes Interview in dem eigenen Schlafzimmer geführt. Im Interview
	wurden die Studienteilnehmenden gebeten, den Forschenden ihre physischen Besitztümer
	in ihrem Zimmer zu zeigen. In der Regel folgte danach das Vorzeigen des
	virtuellen Besitzes, welcher sich meist auf dem Computer oder Smartphone befand 
	\cite{odom2011teenagers}. \\

	Die Interviews wurden mithilfe von Videoaufnahmen, Fotos und Notizen für die spätere Auswertung dokumentiert. Nach der Durchführung
	der Interviews wurden anhand der Aufnahmen und Notizen verschiedene Themen
	definiert. Mithilfe der Themen (Codes) wurden die ausgearbeiteten
	Dokumente codiert. Aus den codierten Dokumenten wurden abschließend Diagramme und Modelle für eine visuelle Darstellung und Auswertung erstellt  \cite{odom2011teenagers}. 

	\section{Diskussion zur Studienmethodik}

	Positive Punkte

	\begin{itemize}
	\item Gute Struktur - Kategosierung in den Findings (Captering emerging themes)
 	\item Interviewort (Schlafzimmer) / Studienumgebung  gut gewählt, da die Besitz im Schlafzimmer ist -> lockere Stimmung
 	\item Viele Auswertungsmethoden (Video, Notizen, Diagramme, Photo und Modelle)
	\end{itemize}

	Bisher kritische Punkte

	\begin{itemize}
	\item Welche Fragen wurden gestellt, wie wurden die Interviews geführt?
	\item Kinder kommen aus sehr ähnlichen Verhältnissen und derselben Stadt = homogene Masse = schlechte Sample Size
	\item Starke Eingrenzung der Probandengruppe = nicht repräsentativ für alle Personen. Diese Limitation ist den Forschern jedoch bekannt.
	\item Forscher gehen sehr interpretativ auf die Aussagen der Probanden ein und schlussfolgern daraus Verbesserungsmöglichkeiten.
	\item Das Paper basiert auf einem "Contextual Inquiry" Studiendesign, da die Umgebung des Schlafzimmers mit begutachted wird und dazu verbale Aussagen der Probanden in Form eines semi-strukturierten Interviews aufgenommen wurden.
	\item Haben auch Männer die Interviews bei Mädchen geführt? Falls ja, ggf. weniger Vertrauen beim Vorzeigen persönlicher Gegenstände
	\item (Nicht definiert bzw. genau gesagt was die genauen Dokumente sind, die nach der Erstellung der Codes, codiert wurde? Sind
		das Transkripierte Interviews, die Notizen?)
		\item Digitale Bestiztümer meist sehr privat – wird nicht alles gezeigt evtl?
		\item Es gab Pilotprojekt. Wie gut wussten Teenager also schon vorher über das Thema bescheid? Beeinflusst?
		\item Interviewlänge - Sind die Interviews nicht zu lange, ist die Länge gerechfertigt?
		\item Videoaufnahme sehr Präsent – vielleicht aufregender bzw. anderes Verhalten als nur bei Audioaufnahme
	\end{itemize}
	

	\clearpage
	\nocite{*}	
	\bibliography{literature}
	\bibliographystyle{abbrv}
\end{document}
