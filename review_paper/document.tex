\documentclass{hsflensburg}

% packages
%-------------------------------------------
\usepackage[utf8]{inputenc}
\usepackage[ngerman]{babel}
\usepackage{csquotes}
\usepackage{placeins}
\usepackage{url}
\usepackage{xcolor}

\MakeOuterQuote{"}
{\renewcommand{\arraystretch}{2.0}% for the vertical padding

\definecolor{orange}{HTML}{FF7F00}

% document title
%-------------------------------------------
\title{Laboraufgabe}
\subtitle{Qualitative Research Methods - Aufgabe 3}

% authors
%-------------------------------------------
\author{
	Michael Frank
}

\begin{document}
	\maketitle
 	 \tableofcontents

  \pagebreak
	
	\section{Rückblick}
	Im Rahmen einer Studie sollte untersucht werden, inwieweit die deutsche Bevölkerung,
	unabhängig des gesetzlichen Rentensystems, finanziell privat vorsorgt. Es werden mehrere Aspekte
	der Altersvorsorge in der deutschen Gesellschaft betrachtet. So wird untersucht, auf welche Möglichkeiten
	der privaten Altersvorsorge die Deutschen zurückgreifen, wie ihr Vertrauen in das deutsche gesetzliche
	Rentensystem ist und wie viele Bürgen im Falle eines Ausfalls bzw. Senkung der Rentenbezüge den Verlust
	bzw. Einschnitt ihrer einzigen finanziellen Einnahmequelle befürchten müssen. \\

	Dafür wurden Probanden aus verschiedenen Altersgruppen bezüglich ihrer privaten Altersvorsorge interviewt. In 
	dieser Ausarbeitung wird der Prozess der Auswertung und Untersuchung der transkribierten Interviews vorgestellt.
 

	\section{Raw Coding}
	Im ersten Schritt, wurden aus den transkribierten Interviews, Codes ausgearbeitet. Es wurden keine Codes 
	im Vorhinein festgelegt, sondern während des ersten Durcharbeitens der Interviews ausgearbeitet. 
	Aufgrund der überschau- baren Anzahl und Länge der Interviews wurden größtenteils einzelne Sätze codiert (mit einem
	Code versehen). Teilweise wurden aber auch mehrere Sätze bzw. Textblöcke unter einem Code zusammengefasst, falls 
	diese anhand des Inhaltes oder der Formulierung (Satzaufbau/-struktur) denselben Code erhalten würden. Für das Eintragen
	der Codes in den Text wurde \textit{Microsoft Word} als Software verwendet. \textit{Word} bietet mithilfe der Kommentar-Funktion 
	eine einfache Möglichkeit Textbausteine zu kommentieren und somit mit einem Code zu versehen. Das Resultat des 
	\textit{Raw Coding Prozesses} kann im Anhang unter \textit{Raw Coding} eingesehen werden. 

	\textbf{Anmerkung}: Aufgrund des häufigen Vorkommens der Begriffe "gesetzliches Rentensystem" und "privater 
	Altersvorsorge" werden diese in den Codes mit "g. R." und "p. A." abgekürzt.

	\section{Axial Coding}

	Nachdem aus den transkribierten Interviews grobe Codes erarbeitet wurden, mussten diese im nächsten Schritt 
	Interviewübergreifend überarbeitet und vereinheitlicht werden, damit bspw. gemeinsame oder auch unterschiedliche
	Meinungen und Erfahrungen zu verschiedenen Themenbereichen einfacher erkannt werden können. Außerdem können
	die Meinungen und Erfahrungen durch die Vergleichsmöglichkeiten unter den Probanden gegebenenfalls erklärt werden. 
	Das Resultat des \textit{Axial Coding Prozesses} kann im Anhang unter \textit{Axial Coding} eingesehen werden.\\

	Die vereinheitlichten und angepassten Codes werden zur Übersicht in dem folgenden Code-System festgehalten:

	\begin{itemize}
	\item \textbf{Code System}
	\begin{itemize}

	\item \textbf{Status:}
	\begin{itemize}
	\item Aktive private Altersvorsorge
	\item Keine aktive private Altersvorsorge
	\item Fehlende Vorsorge bei Frauen
	\end{itemize}

	\item \textbf{Private Altersvorsorgemöglichkeiten:}
	\begin{itemize}
	\item \textbf{Verwendete:}

	\begin{itemize}
	\item Vermögenswirksame Leistungen
	\item Riester-Rente
	\item Wertpapiere
	\item Immobilien
	\item Betriebliche Altersvorsorge
	\item Private Renteneinzahlungen
	\item Lebensversicherung
	\end{itemize}

	\item \textbf{Außerdem bekannt:}
	\begin{itemize}
	\item Erneuerbare Energien
	\item Beteiligungen
	\item Zusatzversicherung
	\item Rürup-Rente
	\end{itemize}


	\end{itemize}

	\item \textbf{Gesetzliches Rentensystem:}
	\begin{itemize}
	\item Notwendigkeit
	\item Vertrauen in Deutschland
	\item Positives Vertrauen
	\item Negative Erwartung
	\item Unsicherheit der Höhe
	\item Rentenunterschiede
	\item Grundrente
	\item Höheres Renteneintrittsalter
	\item Angepasstes Renteneintrittsalter
	\item Unsicherheit Lebensstandard
	\item \textbf{Transparenz:}

	\begin{itemize}
	\item Renteninformationsbrief
	\item Allgemeines Interesse
	\item Positive Rententransparenz
	\item Fehlende Vorsorgebenachrichtigung
	\item Eigeninitiative
	\item Komplexität
	\item Fehlende Eigeninitiative
	\end{itemize}

	\end{itemize}

	\item \textbf{Sicherheit:}
	\begin{itemize}
	\item Grundsicherung
	\item Sozialsystem
	\end{itemize}

	\item \textbf{Probleme}
	\begin{itemize}

	\item \textbf{Geselltschaftliche:}
	\begin{itemize}
	\item Demografischer Wandel
	\item Ungleichheit
	\item Allgemeine Unsicherheit
	\item Familienabhängigkeit
	\item Politik fehlt Realitätsbezug
	\item Altersvorbereitung
	\end{itemize}

	\item \textbf{Gesundheitliche:}
	\begin{itemize}
	\item Altersprobleme
	\item Ängste vor Belastung und Folgen
	\item Altersgerechte Arbeit
	\end{itemize}

	\item \textbf{Private Altersvorsorge:}
	\begin{itemize}
	\item Risiken
	\item Riester-Rente Unzufriedenheit
	\end{itemize}

	\end{itemize}

	\item \textbf{Staatsaufgaben}
	\begin{itemize}
	\item Sparerinitiative
	\item Sparergarantie
	\item Einheitsrente
	\item Rentenprognose
	\end{itemize}

	\item \textbf{Lösungsansätze bei Altersarmut:}
	\begin{itemize}
	\item Arbeiten über Renteneintrittsalter
	\item Kostenreduzierung
	\item Rücklagenverbrauch
	\item Eigenanbau
	\item Familiare Unterstützung
	\item Positive Gehaltsentwicklung
	\end{itemize}

	\end{itemize}
	\end{itemize}

	\pagebreak

	\section{Codebegründungen}

	\textbf{Anmerkung}: Die kursiven Wörter sind die Codes aus dem Code-System. \\

	Eine der wichtigsten Aspekte, welche im Rahmen der Interviews thematisiert wurden, waren, ob der Proband privat für das Alter
	vorsorgt und welche privaten Altersvorsorgemöglichkeiten bekannt und auch gegebenenfalls genutzt werden. Zudem wurden auch 
	das Vertrauen sowie die Erwartungen an das deutsche gesetzliche Rentensystem in den Interviews besprochen. \\

	In den Interviews konnte festgestellt werden, dass einige Probanden bereits Möglichkeiten der privaten Altersvorsorge nutzen,
	andere wiederum nicht \textit{(Aktive private Altersvorsorge, Keine aktive private Altersvorsorge)}. Hier ist anzumerken,
	dass die Probanden, welche Angaben keine private Altersvorsorge zu betreiben, dazu erwähnten, dass Sie sich noch nicht
	mit dem Thema beschäftigt haben. Zudem waren diese Probanden eher jung (zwanzig bis dreißig Jahre alt), was gegebenenfalls
	das fehlende Interesse zu dem Thema erklären kann. \\
	
	Eine Vorsorgemöglichkeit, welche
	überdurchschnittlich oft genannt wurde war die Riester-Rente. Die Riester-Rente ist zwar eine privat finanzierte
	Rente, welche aber mit staatlichen Zulagen gefördert wird, wodurch diese sich möglicherweise in der Gesellschaft besser 
	etabliert hat als andere \textit{(Riester-Rente)} \cite{riester}. Zu der Riester-Renter gab es neutrale bis negative Meinungen, die negativen Meinungen
	wurden im Zusammenhang mit negativen Wertentwicklungen und dem daraus entstehenden Vertrauensverlust genannt
	\textit{(Riester-Rente Unzufriedenheit)}.
	Hier muss zudem angemerkt werden, dass nicht explizit  erfragt wurde, warum oder warum nicht eine Möglichkeit der privaten 					Altersvorsorge genutzt wird. Dadurch hat nicht jeder Proband im speziellen seine Meinung zu den verschiedenen 
	Möglichkeiten genannt, sondern nur wenn dieser es als notwendig empfunden hat. \\

	Bei den Themen Vertrauen und Erwartungen an das gesetzliche Rentensystem ging es in den Interviews deutlich mehr ins Detail.
	Die Codes \textit{Positives Vertrauen} und \textit{Notwendigkeit} kamen in den Interviews häufig vor und thematisieren die 
	Notwendigkeit eines gesetzlichen Rentensystems und das Vertrauen in eine immer bestehende gesetzliche Rente.
	Die Auffassung wurde bspw. damit begründet, dass die finanziell schwächer aufgestellten Bürger nicht privat vorsorgen können und
	somit als Rentner nicht überleben könnten. Neben dem positiven Vertrauen an die Existenz eines gesetzliches Rentensystems
	wurde aber auch die Unsicherheit bei der Höhe der Rentenbezüge und der Verlust eines gewissen Lebensstandards genannt
	\textit{(Negative Erwartung, Untersicherheit der Höhe, Unsicherheit Lebensstandard)}. Ein Grund für die eher pessimistische
	Einstellung gegenüber der Höhe der Rentenbezüge könnte bspw. die zugesendeten Rentenbescheide sein. Diese Bescheide
	wurden in einigen Interviews beschrieben und die dort genannte Höhe wurde teilweise als zu gering angemerkt 
	\textit{(Renteninformationsbrief)}.\\
	
	Auch interessant war die Erwartung, dass das derzeitige Renteneintrittsalter weiter ansteigen wird \textit{(Höheres Renteneintrittsalter)}.
	Dies wurde oftmals im Zusammenhang mit dem demografischen Wandel \textit{(Demografischer Wandel)} beschrieben, was als 
	Ursache für den zu erwartenden Altersanstieg genannt wurde. \\

	Die Transparenz des gesetzlichen Rentensystem wurde in einigen Interviews positiv hervorgehoben \textit{(Positive Rententransparenz)}, 			wobei bspw. die Rentenbescheide als gutes Beispiel für die vorhandenen Transparenz genannt wurde. Außerdem wurde bei der 				Transparenz die notwendige Eigeninitiative bei der Informierung zu den Renteninformationen thematisiert \textit{(Eigeninitiative)}. 				Welche laut eigener Aussage einigen Probanden fehlt \textit{(Fehlende Eigeninitiative)}, obwohl diese das Interesse dafür haben, aber 				aufgrund der Komplexität wenig Motivation finden sich damit auseinanderzusetzen \textit{(Komplexität)}. \\

	Ein weiterer Punkt, welcher im Zusammenhang mit dem höheren zu erwartenden Renteneintrittsalter genannt wurde, war die
	Unsicherheit bzw. Angst vor den körperlichen und geistigen Belastungen beim Arbeiten im höheren Alter 
	\textit{(Ängste vor Belastung und Folgen, Altersprobleme, Altersgerechte Arbeit)}. Diese Gefühle wurden zum Teil
	damit begründet, dass die Arbeitsleistung und Stressresistenz im Alter sinken, aber die Arbeit sich nicht unbedingt
	der Person anpasst und es dabei zu Problemen kommen kann.

	\pagebreak

	\section{Themenbereiche}
	Die Themenbereiche, welche in den Interviews aufkamen, decken, wie anhand des  Code-Systems unter 
	\textit{2. Axial Coding} zu sehen ist, viele unterschiedliche Aspekte des gesetzlichen Rentensystems
	und die Möglichkeiten der privaten Altersvorsorge ab. Folgend werden die für die Forschungsfrage interessanten
	Themenbereiche genannt. Dabei wird sich an den erwähnten Code-System orientiert, weil dieses bereits
	Codes in thematisch ähnliche Blöcke untergliedert.

	\paragraph{Private Altersvorsorge und ihre Möglichkeiten}
	Dieser Themenbereich ist ein essentieller Aspekt für die Studie, da hier mit Probanden darüber gesprochen wurde,
	ob diese aktiv privat Vorsorgen und welche Möglichkeiten Sie dabei nutzen. Der Bereich wurde vor allem
	in den Textbausteinen mit den Codes aus \textit{Status} und \textit{Private Altersvorsorgemöglichkeiten} behandelt.
	
	\paragraph{Erwartungen an das Gesetzliches Rentensystem}
	Auch dieser Themenbereich ist für die Studie sehr interessant, da hier die Notwendigkeit und Sicherheit, sowie
	die zu erwartende Entwicklung der gesetzlichen Rente in Bezug auf die Höhe der Rentenbezüge und des
	Renteneintrittsalters besprochen wurden. 	Vor allem die Codes aus \textit{Gesetzliches Rentensystem}
	sind in diesem Themenbereich enthalten, aber auch Codes zu Gefahren der gesetzlichen Rente wie bspw.
	\textit{Demografischer Wandel} und \textit{Ungleichheit}.

	\paragraph{Lösungsansätze gegen Altersarmut}
	Eine weitere angesprochene Thematik waren die Lösungsansätze gegen eine potentielle Altersarmut, also
	wie die Leute bei einer zu geringen oder gar keiner Rente überleben. Dieses Thema wurde in Texten mit den Codes
	\textit{Lösungsansätze bei Altersartmut} und \textit{Staatsaufgaben} thematisiert.


	\clearpage
	\nocite{*}	
	\bibliography{literature}
	\bibliographystyle{abbrv}
\end{document}
