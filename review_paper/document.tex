\documentclass{hsflensburg}

% packages
%-------------------------------------------
\usepackage[utf8]{inputenc}
\usepackage[ngerman]{babel}
\usepackage{csquotes}
\usepackage{placeins}
\usepackage{url}
\usepackage{xcolor}

\MakeOuterQuote{"}
{\renewcommand{\arraystretch}{2.0}% for the vertical padding

\definecolor{orange}{HTML}{FF7F00}

% document title
%-------------------------------------------
\title{Paper-Review}
\subtitle{Qualitative Research Methods}

% authors
%-------------------------------------------
\author{
	\name{Tom Hartelt}\\
	\institution{Hochschule Flensburg}
	\and
	\name{Martin Hermannsen}\\
	\institution{Hochschule Flensburg}
	\and
	\name{Michael Frank}\\
	\institution{Hochschule Flensburg}
}

\begin{document}
	\maketitle
 	 \tableofcontents

  \pagebreak
	
	\section{Zusammenfassung}
	In der Arbeit \textit{Teenagers and Their Virtual Possessions: Design Opportunities and Issues} von
 	William Odom, John Zimmerman und Jodi Forlizzi aus dem Jahr 2011 wird die Bedeutsamkeit 
	und emotionale Bindung von Jugendlichen an ihren virtuellem Besitz (Musik, Bilder, Nachrichten) untersucht. 
	Im Rahmen der Studie wurden Gespräche mit  Jugendlichen zu ihren physischen und virtuellen Besitztümern 
	geführt. In diesen Gesprächen erhielten die Forscher Einblicke in das alltägliche Leben der 
	Jugendlichen und was Ihnen im  Umgang mit ihren virtuellem Besitz wichtig ist. Außerdem wurde
	wurde in diesen Interviews oftmals der Umgang mit den sozialen Medien wie bspw. Facebook thematisiert, 
	weil viele der erwähnten Besitztümer der Jugendlichen sich auf diesen Plattformen befinden 
	\cite{odom2011teenagers}. \\

	Anhand der Interviews mit den Jugendlichen konnten die Forscher feststellen, dass die 
	virtuellen Besitztümer und sozialen Medien neue Möglichkeiten der Identitätsbildung 
	und -experementierung ermöglichen. Außerdem schlagen die Forscher auf Basis der 
	Erkenntnisse aus den Interviews potentielle Designentscheidungen für die 
	Entwicklungen von neuen Systemen vor, welche den Umgang mit virtuellen 
	Besitztümern verbessern und vereinfachen könnten \cite{odom2011teenagers}. 


	\section{Studienmethodik}
	Für die Studie wurden insgesamt einundzwanzig Kinder und Jugendliche im Alter
	von zwölf bis siebzehn Jahren als Testpersonen rekrutiert. Alle Kinder stammen
	aus einer mittelgroßen Stadt in der USA. Die Probanden gehörten zu Familien
	der gesellschaftlichen Mittel- bzw. Oberschicht und hatten größtenteils Zugang 
	zum Internet \cite{odom2011teenagers}. \\

	Es wurden mit jeder Testperson ein ungefähr neunzig bis einhundertzwanzigminütiges
	Semi-Strukturiertes Interview in dem eigenen Schlafzimmer geführt. Im Interview
	wurden die Studienteilnehmer gebeten, den Forschern ihre physischen Besitztümer
	in ihrem Zimmer zu zeigen. In der Regel folgte danach auch das Vorzeigen des
	virtuellen Besitzes, welches sich auf dem Rechner, Smartphone, usw. befand 
	\cite{odom2011teenagers}. \\

	Die Interviews wurden mithilfe von Videoaufnahmen und Fotos dokumentiert, zudem
	wurden auch Notizen für die spätere Auswertung gemacht. Nach der Durchführung
	der Interviews wurden anhand der Aufnahmen und Notizen verschiedene Themen
	herausgeschrieben. Mithilfe der Themen (Codes) wurden dann die ausgearbeiteten
	Dokumente codiert \cite{odom2011teenagers}.  

	\section{Diskussion zur Studienmethodik}

	Bisher kritische Punkte

	\begin{itemize}
	\item Welche Fragen wurden gestellt, wie wurden die Interviews geführt?
	\item Kinder kommen aus sehr ähnlichen Verhältnissen und derselben Stadt = homogene Masse = schlechte Sample Size
	\item Haben auch Männer die Interviews bei Mädchen geführt? Falls ja, ggf. weniger Vertrauen beim Vorzeigen persönlicher Gegenstände
	\item Nicht definiert bzw. genau gesagt was die genauen Dokumente sind, die nach der Erstellung der Codes, codiert wurde? Sind
		das Transkripierte Interviews, die Notizen?
	\end{itemize}
	

	\clearpage
	\nocite{*}	
	\bibliography{literature}
	\bibliographystyle{abbrv}
\end{document}
